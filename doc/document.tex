\documentclass[]{article}
\usepackage{amsmath, amsthm, amsfonts}
\usepackage{graphics}
\usepackage{float}
\usepackage{epsfig}
\usepackage{amssymb}
\usepackage{dsfont}
\usepackage{latexsym}
\usepackage{newlfont}
\usepackage{epstopdf}
\usepackage{amsthm}
\usepackage{epsfig}
\usepackage{caption}
\usepackage{multirow}
\usepackage[pdftex,breaklinks,colorlinks,linkcolor=black,citecolor=blue,urlcolor=blue]{hyperref}
\usepackage[x11names,table]{xcolor}
\usepackage{graphics}
\usepackage{wrapfig}
\usepackage[rflt]{floatflt}
\usepackage{multicol}
\usepackage{listings} \lstset {language = Python, basicstyle=\bfseries\ttfamily, keywordstyle = \color{blue}, commentstyle = \bf\color{gray}}
\usepackage{tikz}
\usepackage{enumitem}


\newcommand
*
{\itembolasazules}[1]{% bolas 3D
	\footnotesize\protect\tikz[baseline=-3pt]%
	\protect\node[scale=.5, circle, shade, ball
	color=blue]{\color{white}\Large\bf#1};}
%opening
\title{Proyecto de Simulaci\'on}
\author{Karl Lewis Sosa Justiz}


\begin{document}

\begin{figure}
	\maketitle
	\hspace{3,5cm} \includegraphics[width=5cm]{images/índice.jpg} 
\end{figure}


\clearpage
\tableofcontents
\newpage

\section{Orden del Problema}
\subsection{Poblado en Evoluci\'on}
Se dese conocer la evoluci\'on de la poblaci\'on de una determinada regi\'on.
Se conoce que la probabilidad de fallecer de una persona distribuye uniforme
y se corresponde, seg\'un su edad y sexo, con la siguiente tabla:

\begin{table}[htbp]
\begin{center}
\begin{tabular}{c c c }

Edad&Hombre&Mujer  \\
0-12&0.25&0.25 		\\
12-45& 0.1& 0.15  	\\
45-76& 0.3& 0.35    \\
76-125& 0.7& 0.65   \\
\end{tabular}
\end{center}
\end{table}
Del mismo modo, se conoce que la probabilidad de una mujer se embarace
es uniforme y est\'a relacionada con la edad:
\begin{table}[htbp]
	\begin{center}
		\begin{tabular}{c c }
			
			Edad&Probabilidad de Embarazarse \\
			12-15 &0.2		\\
			15-21 &0.45  	\\
			21-35 &0.8   \\
			35-45 &0.4  \\
			45-60 &0.2  \\
			60-125& 0.05  \\
		\end{tabular}
	\end{center}
\end{table}

Para que una mujer quede embarazada debe tener pareja y no haber tenido
el n\'umero m\'aximo de hijos que deseaba tener ella o su pareja en ese momento.
El n\'umero de hijos que cada persona desea tener distribuye uniforme seg\'un la
tabla siguiente :


\begin{table}[htbp]
	\begin{center}
		\begin{tabular}{c c }
			
			N\'umero&Probabilidad \\
			1 &0.6		\\
			2 &0.75  	\\
			3&0.35   \\
			4&0.2  \\
			5 &0.1  \\
			m\'as de 5& 0.05  \\
		\end{tabular}
	\end{center}
\end{table}
\newpage
Para que dos personas sean pareja deben estar solas en ese instante y deben
desear tener pareja. El desear tener pareja est\'a relacionado con la edad:
\begin{table}[htbp]
	\begin{center}
		\begin{tabular}{c c }
			
			Edad&Probabilidad de Querer Pareja \\
			12-15 &0.6		\\
			15-21 &0.65  	\\
			21-35&0.8   \\
			35-45&0.6  \\
			45-60 &0.5  \\
			60-125& 0.2  \\
		\end{tabular}
	\end{center}
\end{table}

Si dos personas de diferente sexo est\'an solas y ambas desean querer tener
parejas entonces la probabilidad de volverse pareja est\'a relacionada con la diferencia de edad:
\begin{table}[htbp]
	\begin{center}
		\begin{tabular}{c c }
			
			Diferencia de Edad&Probabilidad de Establecer Pareja \\
			0-5 &0.45		\\
			5-10 &0.4 	\\
			10-15&0.35   \\
			15-20&0.25  \\
			20 o m\'as &0.15  \\
		\end{tabular}
	\end{center}
\end{table}

Cuando dos personas est\'an en pareja la probabilidad de que ocurra una ruptura distribuye uniforme y es de 0.2. Cuando una persona se separa, o enviuda,
necesita estar sola p or un per\'iodo de tiempo que distribuye exponencial con un
par\'ametro que est\'a relacionado con la edad:
\begin{table}[htbp]
	\begin{center}
		\begin{tabular}{c c }
			
			Edad & $\lambda$ \\
			12-15 & 3 meses		\\
			15-21 & 6 meses 	\\
			21-35&  6 meses  \\
			35-45& 1 a\~nos  \\
			45-60 &2 a\~nos  \\
			60-125& 4 a\~nos
		\end{tabular}
	\end{center}
\end{table}

\newpage
Cuando est\'an dadas todas las condiciones y una mujer queda embarazada
puede tener o no un embarazo m\'ultiple y esto distribuye uniforme acorde a las
probabilidades siguientes:
\begin{table}[htbp]
	\begin{center}
		\begin{tabular}{c c }
			
			N\'umero de Beb\'es & Probabilidad \\
			1 & 0.7	\\
			2 & 0.18 	\\
			3&  0.08  \\
			4& 0.04  \\
			5&0.02  \\

		\end{tabular}
	\end{center}
\end{table}

La probabilidad del sexo de cada beb\'e nacido es uniforme 0,5.
Asumiendo que se tiene una poblaci\'on inicial de M mujeres y H hombres y
que cada poblador, en el instante incial, tiene una edad que distribuye uniforme
(U(0,100)). Realice un pro ceso de simulaci\'on para determinar como evoluciona
la poblaci\'on en un per\'io do de 100 a\~nos.

\section{Principales Ideas seguidas para la soluci\'on del problema}
\begin{itemize}
	\item Se va a analizar el proceso con intervalos mensuales
	\item Se tiene una clase Person que va se va actualizando cada mes seg\'un su estado 
	\item Se tiene una clase Population que tiene el r\'ecord de todas las personas y maneja el estado de las parejas
\end{itemize}
\newpage
\section{Modelo de Simulaci\'on de Eventos Discretos desarrollado para resolver el problema}
\begin{lstlisting}
	while time!=0:
		#one_more_month lo que realmente hace
		#es actualizar la edad, si quiere estar
		#en pareja, el tiempo de luto y el
		#tiempo de embarazo 
		for i in men:
			i.one_more_month()
		for i in women:
			i.one_more_month()
		#Se elimina los muertos en ese periodo
		remove_dead_people()
		#Actualiza el tiempo que le resta a las parejas
		update_couples()
		#Actualiza el tiempo restante a los embarazos
		update_kids()
		#Agregar nuevos nacidos a la poblacion
		give_birth()
		#Por cada pareja que cumpla los requisitos 
		#poner a la mujer a gestar
		make_kids()
		#Eliminar parejas que ya cumplieron su tiempo
		remove_couples()
		#Formar nuevas parejas segun las condiciones
		make_couples()
		#Paso de un mes
		time-=1

\end{lstlisting}
\section{Consideraciones obtenidas a partir de la ejecuci\'on de las simulaciones del problema}
Cuando se inicia la poblaci\'on una uniforme no es lo ideal para generar las edades ya que es asi como se comportan las edades en una poblaci\'on una distribuci\'on normal ser\'ia m\'as efectiva. Adem\'as al formar pareja no es factible comprobar cada par de personas y ver si cumplen las condiciones ya que para poblaciones del orden de $10^{5}$ este proceso al ser de orden cuadr\'atico hace imposible su calculo, en cambio asignarle a cada mujer una pareja (d\'andole  m\'as opotunidades a los que tienen edades cercanas que son los que m\'as posibilidades tienen) permitir\'ia analizar mayor cantidad de personas.
Por otra parte el en tiempo de luto el $\lambda$ fue tomado como valor esperado (1/$\lambda$) ya que est\'a dado en meses y tiene m\'as sentido. En el transcurso de 100 a\~nos las poblaciones crecen lo que todav\'ia no se puede ver el boom del crecimiento exponencial ya que solo hay 5 o 6 generaciones a lo sumo.
\section{Enlace al repositorio del proyecto en Github}

\end{document}
